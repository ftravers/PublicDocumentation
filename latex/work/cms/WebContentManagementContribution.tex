\subsection{WCM Contribution}

The first and most basic task will be the simple editing of a web
page.  Lets go to the Japan Home page and go into contribution mode.

\begin{verbatim}
http://my.oracle.com/site/japan/index.htm

Ctrl-Shift-F5
\end{verbatim}

This puts you in `contribution' mode.  You then have different types
of elements you can edit.  Typically these will be broken down into
your traditional WYSIWYG editor as can be seen in the figure: \emph{WYSIWYG Editor
}, or a dynamic list.

\subsection{Contribution}

\subsubsection{Dynamic Lists}

Please see the definitions sections for an explanation of Dynamic Lists.

\subsubsection{Linking to Documents in WYSIWYG}

When you want to link to another page/document in content server
always use the `Link To' icon in contribution mode.  Do not hardcode
the link by hand.  The reason for this is that codes get inserted into
the document that handle resolving to the correct host name, when the
document gets published from contribution to consumption.

\subsection{Desktop Integration}

Desktop integration is an excellent tool to assist people who would
like to contribute documents to the website.  It modifies the behavior
of Windows Explorer to allow a `shared' drive for the Oracle Content
Server.  Users then use the famliar concept of a folder hierarchy to
sort their content and they can update the metadata from within
Windows Explorer as well.

When you have several documents/files to contribute, then you can
simply drag and drop the files from your desktop into the Content
Server folder, avoiding the need to check in files one-by-one.

You can get Desktop Integration by going to MyDesktop and installing
the `Oracle Desktop Integration' software application.

\subsection{Site Manager}

Site manager allows assigned users to \emph{add/edit/rearrange/delete} nodes
from the site hierarchy.  They can specify which templates will be
used for a given section.  This is a higher level of privilege than
your normal contributors themselves.